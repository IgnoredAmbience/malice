\documentclass[a4paper,11pt]{article}
\usepackage{fullpage, titling, amsmath, footnote}
\makesavenoteenv{tabular}
\setlength{\droptitle}{-50pt}
\setlength{\parskip}{0.3cm}
\setlength{\parindent}{0cm}

% This is the preamble section where you can include extra packages etc.

\begin{document}

\title{MAlice Language Specification}

\author{Ethel Bardsley \and Joe Slade \and Thomas Wood}

\date{\today}         % inserts today's date

\maketitle            % generates the title from the data above

\section{BNF Grammar} 
\begin{tabular}{lcl}
Program     & $\to$ & Statements \\
Statements  & $\to$ & Statement Terminator Statements $| \epsilon$ \\
Terminator  & $\to$ & `\verb:,:' $|$ `\verb:.:' $|$ `\verb:and:' $|$ `\verb:but:' $|$ `\verb:then:' $|$ `\verb|?|'\\
\\
Statement   & $\to$ & \emph{Id} `\verb|was|' `\verb|a|' Type Too \\
            &  $|$  & \emph{Id} `\verb|had|' Exp Type \\
            &  $|$  & Variable `\verb|became|' Exp \\
            &  $|$  & Variable `\verb|ate|' \\
            &  $|$  & Variable `\verb|drank|' \\
            &  $|$  & `\verb|what|' `\verb|was|' Variable \\
            &  $|$  & String `\verb|thought|' `\verb|Alice|' \\
            &  $|$  & Printable `\verb|said|' `\verb|Alice|' \\
            &  $|$  & Printable `\verb|spoke|' \\
            &  $|$  & `\verb:Alice:' `\verb:found:' Exp \\
            &  $|$  & `\verb|eventually|' `\verb|(|' Exp `\verb|)|' `\verb|because|' Statements `\verb|enough|' `\verb|times|' \\
            &  $|$  & `\verb|perhaps|' `\verb|(|' Exp `\verb|)|' `\verb|so|' Statements EndIf \\
ElseIf      & $\to$ & `\verb|maybe|' `\verb|(|' Exp `\verb|)|' `\verb|so|' Statements EndIf $|$ Statements EndIf \\
EndIf       & $\to$ & `\verb|or|' ElseIf $|$ `\verb|Alice was unsure which|' \\
\\
Type        & $\to$ & `\verb|number|' $|$ `\verb|letter|' $|$ `\verb|sentence|' \\
Too         & $\to$ & `\verb|too|' $|$ $\epsilon$ \\
\\
Exp         & $\to$ & Exp `\verb:||:' Exp1 $|$ Exp1 \\
Exp1        & $\to$ & Exp1 `\verb:&&:' Exp2 $|$ Exp2 \\
Exp2        & $\to$ & Exp2 `\verb:|:' Exp3 $|$ Exp3 \\
Exp3        & $\to$ & Exp3 `\verb:^:' Exp4 $|$ Exp4 \\
Exp4        & $\to$ & Exp4 `\verb:&:' Exp5 $|$ Exp5 \\
Exp5        & $\to$ & Exp5 (`\verb:==:' $|$ `\verb:!=:') Exp6 $|$ Exp6 \\
Exp6        & $\to$ & Exp6 (`\verb:<:' $|$ `\verb:<=:' $|$ `\verb:>:' $|$ `\verb:>=:') Exp7 $|$ Exp7 \\
Exp7        & $\to$ & Exp7 (`\verb:+:' $|$ `\verb:-:') Exp8 $|$ Exp8 \\
Exp8        & $\to$ & Exp8 (`\verb:*:' $|$ `\verb:/:' $|$ `\verb:%:') Exp9 $|$ Exp9 \\
Exp9        & $\to$ & `\verb:~:' Exp9 $|$ Val \\
\\
Val         & $\to$ & Variable $|$ \emph{Int} $|$ Character $|$ String \\
Variable    & $\to$ & \emph{Id} $|$ \emph{Id} `\verb|'s|' Exp `\verb|piece|' \\
Character   & $\to$ & `\verb|'|' \emph{Char} `\verb|'|' \\
String      & $\to$ & `\verb|"|' Str `\verb|"|' \\
Str         & $\to$ & \emph{Char} Str $|$ $\epsilon$ 
\end{tabular}

\begin{itemize}
\item \emph{Int} is an integer, matching the regular expression pattern \verb:[0-9]+:
\item \emph{Id} is a variable identifier, matching \verb:[a-zA-Z_]+:
\item \emph{Char} is any ASCII character
\end{itemize}

\section{Semantics}
\subsection{Types}
\subsubsection{Number}
Numbers are signed integers of length 32 bits. Underflow and overflow are
undefined behaviours. All operators listed in the operators section can be
used.

However, when returned via the program's exit code, all numbers will be
returned as unsigned 8 bit integers, from the lowest 8 bits of the number.

\subsubsection{Letter}
Although \verb:letter: appears as a type in the given examples, there are no
working exapmles in which its functionality is exhibited. Consequently, nothing
can be inferred about this possible type, including whether it is a valid type
or not!

However, it was stated at a later date that the letter type is to be stored as
an 8-bit ASCII value. This means that it is a valid type. The BNF above has been
modified to include it.

The given examples imply that it is not permitted to mix number and letter types
over a single operation. Nothing can be inferred about how the letter type should
be returned, nor what operators it is compatible with.

\subsubsection{Sentences}
This is a string type, again, nothing can be inferred about them.

\subsubsection{Arrays}
TODO

\subsection{Statements}
An Alice program is defined as a list of statements.

\subsubsection{Output}
The \verb:Alice found: statement is analogous to the return statement of
other languages. It evaluates its parameter (an expression) and returns the
value.

For example:\\
\verb:  Alice found 3.:\\
returns the value 3.

When returning from the program as a whole, it returns the value of the expression
using the program's termination code.

\subsubsection{Declaration}
The \verb:was a: statement declares the preceding identifier as a variable of
the given type.

Declaring the same variable name multiple times is not permitted and will
result in a compile-time error.

The keyword \verb:too: may be placed at the end of this statement. No meaning
could be inferred from the examples given, so none has been assumed at this point.

For example:\\
\verb:  x was a number:\\
declares a variable called $x$ as a number

\subsubsection{Assignment}
The \verb:became: statement assigns the value of an expression to the given
variable.

The type of the expression must match the type of the variable, otherwise a
compile-time error will result.

For example:\\
\verb:  x became 5:\\
assigns 5 to $x$.

\subsubsection{Increment and Decrement}
The \verb:drank: statement decrements the given variable by 1.

The \verb:ate: statement increments the given variable by 1.

For example:\\
\verb:  x drank:\\
if $x$ is 5, $x$ will become 4

For example:\\
\verb:  x ate:\\
if $x$ is 5, $x$ will become 6

\subsubsection{Control Flow}
TODO

\subsection{Expressions}
TO UPDATE
\begin{tabular}{ccc}
Operator & Operation   & Precedence \\
\hline
\hline
\verb:|: & Bitwise OR  & 1 \\
\hline
\verb:&: & Bitwise XOR & 2 \\
\hline
\verb:^: & Bitwise AND & 3 \\
\hline
\verb:+: & Addition    & 4 \\
\verb:-: & Subtraction & 4 \\
\hline
\verb:*: & Multiplication & 5 \\
\verb:/: & Division       & 5 \\
\verb:%: & Modulo         & 5 \\
\hline
\verb:~: & Bitwise NOT    & 6 \\
\end{tabular}

\begin{itemize}
\item Numerically higher precedences bind more tightly.

\item All operators are mathematically associative, and implemented as
left-associative.

\item Division by 0 is undefined and will be handled by the operating system.

\item All operators are binary, except for Bitwise NOT which is unary.

\item All operator precedences (except for Bitwise NOT) taken from example
programs given, including the additional ones provided after the Milestone 1
deadline. Bitwise NOT presumed to have the highest precedence, since it is the
only Unary Operator.

\item Underflow and overflow conditions are undefined.
\end{itemize}

\end{document}
