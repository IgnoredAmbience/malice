\documentclass[a4wide, 11pt]{article}
\usepackage{a4}
\setlength{\parskip}{0.3cm}
\setlength{\parindent}{0cm}

% This is the preamble section where you can include extra packages etc.

\begin{document}

\title{MAlice Language Specification}

\author{Ethel Bardsley \and Joe Slade \and Thomas Wood}

\date{\today}         % inserts today's date

\maketitle            % generates the title from the data above

\section{BNF Grammar} 
\begin{tabular}{lcl}
Program     & $\to$ & Statements Output \\
Statements  & $\to$ & Statement Terminator Statements $|$ $\epsilon$ \\
Terminator  & $\to$ & \verb:,: $|$ \verb:.: $|$ \verb:and: $|$ \verb:but: $|$ \verb:then: \\
Output      & $\to$ & \verb:Alice found: Exp \verb:.: \\
\\
Statement   & $\to$ & \emph{Id} \verb|was a| Type Too \\
            &  $|$  & \emph{Id} \verb|became| Exp \\
            &  $|$  & \emph{Id} \verb|ate| \\
            &  $|$  & \emph{Id} \verb|drank| \\
\\
Type        & $\to$ & \verb|number| \\
Too         & $\to$ & \verb|too| $|$ $\epsilon$ \\
Exp         & $\to$ & Exp \verb:|: Exp1 \\
            &  $|$  & Exp \verb:^: Exp1 \\
            &  $|$  & Exp \verb:&: Exp1 \\
            &  $|$  & Exp1 \\
\\
Exp1        & $\to$ & Exp1 \verb:+: Exp2 \\
            &  $|$  & Exp2 \\
\\
Exp2        & $\to$ & Exp2 \verb:*: Exp3 \\
            &  $|$  & Exp2 \verb:/: Exp3 \\
            &  $|$  & Exp2 \verb:%: Exp3 \\
            &  $|$  & Exp3 \\
\\
Exp3        & $\to$ & \verb:~:Val $|$ Val \\
Val         & $\to$ & \emph{Int} $|$ \emph{Id} \\
\end{tabular}

\begin{itemize}
\item \emph{Int} is an integer, matching the regular expression pattern \verb:[0-9]+:
\item \emph{Id} is a variable identifier, matching \verb:[a-zA-Z_]+:
\end{itemize}

\section{Semantics}
\subsection{Types}
\subsubsection{Number}
Numbers are unsigned integers of length 8 bits (ie: they can hold the range
0-255). Furthermore underflow and overflow are undefined behaviours. All
operators listed in the operators section can be used.

\subsubsection{Letter}
Although \verb:letter: appears as a type in the given examples, there are no
working exapmles in which its functionality is exhibited. Consequently, nothing
can be inferred about this possible type, including whether it is a valid type
or not!

As such, it is not included in this version of the language specification.

\subsection{Statements}
An Alice program is defined as a list of statements followed by the output
statement.

\subsubsection{Output}
The \verb:Alice found: statement is analogous to the return statement of
other languages. It evaluates its parameter (an expression) and returns the
value.

For example:\\
\verb:  Alice found 3.:\\
returns the value 3.

\subsubsection{Declaration}
The \verb:was a: statement declares the preceding identifier as a variable of
the given type.

Declaring the same variable name multiple times is not permitted and will
result in a compile-time error.

For example:\\
\verb:  x was a number:\\
declares a variable called $x$ as a number

\subsubsection{Assignment}
The \verb:became: statement assigns the value of an expression to the given
variable.

The type of the expression must match the type of the variable, otherwise a
compile-time error will result.

For example:\\
\verb:  x became 5:\\
assigns 5 to $x$.

\subsubsection{Increment and Decrement}
The \verb:drank: statement decrements the given variable by 1.

The \verb:ate: statement increments the given variable by 1.

For example:\\
\verb:  x drank:\\
if $x$ is 5, $x$ will become 4

For example:\\
\verb:  x ate:\\
if $x$ is 5, $x$ will become 6

\subsection{Expressions}
\begin{tabular}{ccc}
Operator & Operation   & Precedence \\
\hline
\hline
\verb:|: & Bitwise OR  & 1 \\
\verb:^: & Bitwise XOR & 1 \\
\verb:&: & Bitwise AND & 1 \\
\hline
\verb:+: & Addition    & 2 \\
\hline
\verb:*: & Multiplication & 3 \\
\verb:/: & Division       & 3 \\
\verb:%: & Modulo         & 3 \\
\hline
\verb:~: & Bitwise NOT    & 4 \\
\end{tabular}

\begin{itemize}
\item Numerically higher precedences bind more tightly.

\item All operators are mathematically associative, and implemented as
left-associative.

\item Division by 0 is undefined and will be handled by the operating system.

\item All operators are binary, except for Bitwise NOT which is unary.

\item The only precedences that were determinable from the example files given
were those for the \verb|+| and \verb|*| operators. All other precedences have
been taken from the common usage, or where none is obvious, from the C language.
\end{itemize}

\end{document}
